\documentclass{article}

\usepackage[russian]{babel}
\usepackage[T2A]{fontenc}
\usepackage[utf8]{inputenc}
\usepackage{authblk}
\usepackage[T1]{fontenc}

\title{Code Style}
\author{Егоренко Н.И. ИА-031}
\affil{email: kotkit555@gmail.com,  github: @FrenkyFix}
\date{Февраль 2022}

\begin{document}

\maketitle

\section{C/C++}
\subsection{Отступы}
В начале строки отступ равен 4 пробелам. Отступ между ключевым словом и открывающей скобкой, отступ после знаков препинаний, отступ перед оператором сравнения и после него равны 1 пробелу:\\\\
\newcommand*{\Tab}{\hspace*{0.5cm}}
\begin{center}
\begin{tabular}{ | l | }
\hline
\\
int main() \{\\
\Tab int number;\\
\Tab if (number);\\
\Tab for (int i = 0; i < 9; i++) \{\\
\Tab\Tab ...;\\
\Tab \}\\
\}\\\\
\hline
\end{tabular}
\end{center}
\newpage
\subsection{Библиотеки и константы}
Библиотеки и константы отделены от функций пустой строкой. Сами библиотеки и константы тоже разделены пустой строкой:
\begin{center}
\begin{tabular}{ | l | }
\hline
\\
\#include <iostream> \\\\
\#define MAX = 5 \\\\
int main() \{\\
\Tab ...\\
\}\\\\
\hline
\end{tabular}
\end{center}
\\\\
\subsection{Функции}
В объявлении функций стоит учитывать последовательность применения этих функций, чтобы избежать ошибок.\\
Между функциями ставится пустая строка.\\
Закрывающая скобка ставится на том же уровне, что и начало функции, к которой она принадлежит:\\
\begin{center}
\begin{tabular}{ | l | }

\hline
\\
void do\_something(double number) \{\\
    \Tab ...\\
\}\\\\

void print\_something(double number) \{\\
    \Tab ...\\
\}\\\\

void program(string *buffer) \{\\
    \Tab ...\\
	\Tab do\_something(number);\\
	\Tab ...\\
	\Tab print\_something(number);\\
	\Tab ...\\

\}\\
\\
\hline

\end{tabular}
\end{center}
\\\\
\newpage
\subsection{Циклы}\\
Оператор и параметры цикла разделяются пробелом. Для инициализации счетчика циклов используются переменные i, j, k: \\\\
\large \textbf{for}\\\\
\normalsize
\begin{left}
\begin{tabular}{ | l | }

\hline
\\
for (int i = 0; i < 3; i++) \{\\
         \Tab for (int j = 0; j < 3; j++) \{\\
               \Tab\Tab ...; \\
           \Tab \} \\
\}
\\\\
\hline

\end{tabular}
\end{left}
\\\\
\\\large \textbf{while}\\\\
\normalsize
\begin{left}
\begin{tabular}{ | l | }

\hline
\\
while (i < 3) \{\\
               \Tab ...; \\
\}
\\\\
\hline

\end{tabular}
\\\\
\end{left}\\\\
\\\large \textbf{do while}\\\\
\begin{left}
\begin{tabular}{ | l | }

\hline
\\
do \{\\
               \Tab ...; \\
        \} while (i < 3);
\\\\
\hline

\end{tabular}
\\\\
\end{left}
\newpage
\subsection{switch}
Перед каждым вариантом ставится отступ в 4 пробела. После ключевого слова переход на новую строку с двойным отступом вначале:\\
\begin{center}
\begin{tabular}{ | l | }

\hline
\\
switch (number) \{\\
    \Tab case 1:\\
        \Tab\Tab cout << \verb|"|first\verb|"| << endl;\\ 
        \Tab\Tab break;\\
    \Tab case 2:\\
        \Tab\Tab cout << \verb|"|second\verb|"| << endl;\\ 
        \Tab\Tab break;\\
\}
\\\\
\hline

\end{tabular}
\end{center}
\\\\
\subsection{if}
if и else начинаются с новых строк. Если в условии только одно действие, то оно пишется на строке с if/else.\\
\begin{center}
\begin{tabular}{ | l | }

\hline
\\
\Tab if (a < b) \{\\
    \Tab\Tab a += 3;\\
    \Tab\Tab b += 5;\\
\Tab \}\\
\Tab if (a >= b) \Tab return a;\\
\Tab else if (b >= 12) \Tab return b;\\
\\
\hline

\end{tabular}
\end{center}
\\\\
\newpage
\subsection{Структуры}
Структуры объявляются с помощью typedef для удобства. Названия структур пишутся с заглавной буквы:\\
\begin{center}
\begin{tabular}{ | l | }

\hline
\\typedef struct Node \{\\
    \Tab ...\\
\} Node;
\\\\
\hline

\end{tabular}
\end{center}

\end{document}